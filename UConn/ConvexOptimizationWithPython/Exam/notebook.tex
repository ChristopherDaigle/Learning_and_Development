
% Default to the notebook output style

    


% Inherit from the specified cell style.




    
\documentclass[11pt]{article}

    
    
    \usepackage[T1]{fontenc}
    % Nicer default font (+ math font) than Computer Modern for most use cases
    \usepackage{mathpazo}

    % Basic figure setup, for now with no caption control since it's done
    % automatically by Pandoc (which extracts ![](path) syntax from Markdown).
    \usepackage{graphicx}
    % We will generate all images so they have a width \maxwidth. This means
    % that they will get their normal width if they fit onto the page, but
    % are scaled down if they would overflow the margins.
    \makeatletter
    \def\maxwidth{\ifdim\Gin@nat@width>\linewidth\linewidth
    \else\Gin@nat@width\fi}
    \makeatother
    \let\Oldincludegraphics\includegraphics
    % Set max figure width to be 80% of text width, for now hardcoded.
    \renewcommand{\includegraphics}[1]{\Oldincludegraphics[width=.8\maxwidth]{#1}}
    % Ensure that by default, figures have no caption (until we provide a
    % proper Figure object with a Caption API and a way to capture that
    % in the conversion process - todo).
    \usepackage{caption}
    \DeclareCaptionLabelFormat{nolabel}{}
    \captionsetup{labelformat=nolabel}

    \usepackage{adjustbox} % Used to constrain images to a maximum size 
    \usepackage{xcolor} % Allow colors to be defined
    \usepackage{enumerate} % Needed for markdown enumerations to work
    \usepackage{geometry} % Used to adjust the document margins
    \usepackage{amsmath} % Equations
    \usepackage{amssymb} % Equations
    \usepackage{textcomp} % defines textquotesingle
    % Hack from http://tex.stackexchange.com/a/47451/13684:
    \AtBeginDocument{%
        \def\PYZsq{\textquotesingle}% Upright quotes in Pygmentized code
    }
    \usepackage{upquote} % Upright quotes for verbatim code
    \usepackage{eurosym} % defines \euro
    \usepackage[mathletters]{ucs} % Extended unicode (utf-8) support
    \usepackage[utf8x]{inputenc} % Allow utf-8 characters in the tex document
    \usepackage{fancyvrb} % verbatim replacement that allows latex
    \usepackage{grffile} % extends the file name processing of package graphics 
                         % to support a larger range 
    % The hyperref package gives us a pdf with properly built
    % internal navigation ('pdf bookmarks' for the table of contents,
    % internal cross-reference links, web links for URLs, etc.)
    \usepackage{hyperref}
    \usepackage{longtable} % longtable support required by pandoc >1.10
    \usepackage{booktabs}  % table support for pandoc > 1.12.2
    \usepackage[inline]{enumitem} % IRkernel/repr support (it uses the enumerate* environment)
    \usepackage[normalem]{ulem} % ulem is needed to support strikethroughs (\sout)
                                % normalem makes italics be italics, not underlines
    

    
    
    % Colors for the hyperref package
    \definecolor{urlcolor}{rgb}{0,.145,.698}
    \definecolor{linkcolor}{rgb}{.71,0.21,0.01}
    \definecolor{citecolor}{rgb}{.12,.54,.11}

    % ANSI colors
    \definecolor{ansi-black}{HTML}{3E424D}
    \definecolor{ansi-black-intense}{HTML}{282C36}
    \definecolor{ansi-red}{HTML}{E75C58}
    \definecolor{ansi-red-intense}{HTML}{B22B31}
    \definecolor{ansi-green}{HTML}{00A250}
    \definecolor{ansi-green-intense}{HTML}{007427}
    \definecolor{ansi-yellow}{HTML}{DDB62B}
    \definecolor{ansi-yellow-intense}{HTML}{B27D12}
    \definecolor{ansi-blue}{HTML}{208FFB}
    \definecolor{ansi-blue-intense}{HTML}{0065CA}
    \definecolor{ansi-magenta}{HTML}{D160C4}
    \definecolor{ansi-magenta-intense}{HTML}{A03196}
    \definecolor{ansi-cyan}{HTML}{60C6C8}
    \definecolor{ansi-cyan-intense}{HTML}{258F8F}
    \definecolor{ansi-white}{HTML}{C5C1B4}
    \definecolor{ansi-white-intense}{HTML}{A1A6B2}

    % commands and environments needed by pandoc snippets
    % extracted from the output of `pandoc -s`
    \providecommand{\tightlist}{%
      \setlength{\itemsep}{0pt}\setlength{\parskip}{0pt}}
    \DefineVerbatimEnvironment{Highlighting}{Verbatim}{commandchars=\\\{\}}
    % Add ',fontsize=\small' for more characters per line
    \newenvironment{Shaded}{}{}
    \newcommand{\KeywordTok}[1]{\textcolor[rgb]{0.00,0.44,0.13}{\textbf{{#1}}}}
    \newcommand{\DataTypeTok}[1]{\textcolor[rgb]{0.56,0.13,0.00}{{#1}}}
    \newcommand{\DecValTok}[1]{\textcolor[rgb]{0.25,0.63,0.44}{{#1}}}
    \newcommand{\BaseNTok}[1]{\textcolor[rgb]{0.25,0.63,0.44}{{#1}}}
    \newcommand{\FloatTok}[1]{\textcolor[rgb]{0.25,0.63,0.44}{{#1}}}
    \newcommand{\CharTok}[1]{\textcolor[rgb]{0.25,0.44,0.63}{{#1}}}
    \newcommand{\StringTok}[1]{\textcolor[rgb]{0.25,0.44,0.63}{{#1}}}
    \newcommand{\CommentTok}[1]{\textcolor[rgb]{0.38,0.63,0.69}{\textit{{#1}}}}
    \newcommand{\OtherTok}[1]{\textcolor[rgb]{0.00,0.44,0.13}{{#1}}}
    \newcommand{\AlertTok}[1]{\textcolor[rgb]{1.00,0.00,0.00}{\textbf{{#1}}}}
    \newcommand{\FunctionTok}[1]{\textcolor[rgb]{0.02,0.16,0.49}{{#1}}}
    \newcommand{\RegionMarkerTok}[1]{{#1}}
    \newcommand{\ErrorTok}[1]{\textcolor[rgb]{1.00,0.00,0.00}{\textbf{{#1}}}}
    \newcommand{\NormalTok}[1]{{#1}}
    
    % Additional commands for more recent versions of Pandoc
    \newcommand{\ConstantTok}[1]{\textcolor[rgb]{0.53,0.00,0.00}{{#1}}}
    \newcommand{\SpecialCharTok}[1]{\textcolor[rgb]{0.25,0.44,0.63}{{#1}}}
    \newcommand{\VerbatimStringTok}[1]{\textcolor[rgb]{0.25,0.44,0.63}{{#1}}}
    \newcommand{\SpecialStringTok}[1]{\textcolor[rgb]{0.73,0.40,0.53}{{#1}}}
    \newcommand{\ImportTok}[1]{{#1}}
    \newcommand{\DocumentationTok}[1]{\textcolor[rgb]{0.73,0.13,0.13}{\textit{{#1}}}}
    \newcommand{\AnnotationTok}[1]{\textcolor[rgb]{0.38,0.63,0.69}{\textbf{\textit{{#1}}}}}
    \newcommand{\CommentVarTok}[1]{\textcolor[rgb]{0.38,0.63,0.69}{\textbf{\textit{{#1}}}}}
    \newcommand{\VariableTok}[1]{\textcolor[rgb]{0.10,0.09,0.49}{{#1}}}
    \newcommand{\ControlFlowTok}[1]{\textcolor[rgb]{0.00,0.44,0.13}{\textbf{{#1}}}}
    \newcommand{\OperatorTok}[1]{\textcolor[rgb]{0.40,0.40,0.40}{{#1}}}
    \newcommand{\BuiltInTok}[1]{{#1}}
    \newcommand{\ExtensionTok}[1]{{#1}}
    \newcommand{\PreprocessorTok}[1]{\textcolor[rgb]{0.74,0.48,0.00}{{#1}}}
    \newcommand{\AttributeTok}[1]{\textcolor[rgb]{0.49,0.56,0.16}{{#1}}}
    \newcommand{\InformationTok}[1]{\textcolor[rgb]{0.38,0.63,0.69}{\textbf{\textit{{#1}}}}}
    \newcommand{\WarningTok}[1]{\textcolor[rgb]{0.38,0.63,0.69}{\textbf{\textit{{#1}}}}}
    
    
    % Define a nice break command that doesn't care if a line doesn't already
    % exist.
    \def\br{\hspace*{\fill} \\* }
    % Math Jax compatability definitions
    \def\gt{>}
    \def\lt{<}
    % Document parameters
    \title{DaigleExam}
    
    
    

    % Pygments definitions
    
\makeatletter
\def\PY@reset{\let\PY@it=\relax \let\PY@bf=\relax%
    \let\PY@ul=\relax \let\PY@tc=\relax%
    \let\PY@bc=\relax \let\PY@ff=\relax}
\def\PY@tok#1{\csname PY@tok@#1\endcsname}
\def\PY@toks#1+{\ifx\relax#1\empty\else%
    \PY@tok{#1}\expandafter\PY@toks\fi}
\def\PY@do#1{\PY@bc{\PY@tc{\PY@ul{%
    \PY@it{\PY@bf{\PY@ff{#1}}}}}}}
\def\PY#1#2{\PY@reset\PY@toks#1+\relax+\PY@do{#2}}

\expandafter\def\csname PY@tok@w\endcsname{\def\PY@tc##1{\textcolor[rgb]{0.73,0.73,0.73}{##1}}}
\expandafter\def\csname PY@tok@c\endcsname{\let\PY@it=\textit\def\PY@tc##1{\textcolor[rgb]{0.25,0.50,0.50}{##1}}}
\expandafter\def\csname PY@tok@cp\endcsname{\def\PY@tc##1{\textcolor[rgb]{0.74,0.48,0.00}{##1}}}
\expandafter\def\csname PY@tok@k\endcsname{\let\PY@bf=\textbf\def\PY@tc##1{\textcolor[rgb]{0.00,0.50,0.00}{##1}}}
\expandafter\def\csname PY@tok@kp\endcsname{\def\PY@tc##1{\textcolor[rgb]{0.00,0.50,0.00}{##1}}}
\expandafter\def\csname PY@tok@kt\endcsname{\def\PY@tc##1{\textcolor[rgb]{0.69,0.00,0.25}{##1}}}
\expandafter\def\csname PY@tok@o\endcsname{\def\PY@tc##1{\textcolor[rgb]{0.40,0.40,0.40}{##1}}}
\expandafter\def\csname PY@tok@ow\endcsname{\let\PY@bf=\textbf\def\PY@tc##1{\textcolor[rgb]{0.67,0.13,1.00}{##1}}}
\expandafter\def\csname PY@tok@nb\endcsname{\def\PY@tc##1{\textcolor[rgb]{0.00,0.50,0.00}{##1}}}
\expandafter\def\csname PY@tok@nf\endcsname{\def\PY@tc##1{\textcolor[rgb]{0.00,0.00,1.00}{##1}}}
\expandafter\def\csname PY@tok@nc\endcsname{\let\PY@bf=\textbf\def\PY@tc##1{\textcolor[rgb]{0.00,0.00,1.00}{##1}}}
\expandafter\def\csname PY@tok@nn\endcsname{\let\PY@bf=\textbf\def\PY@tc##1{\textcolor[rgb]{0.00,0.00,1.00}{##1}}}
\expandafter\def\csname PY@tok@ne\endcsname{\let\PY@bf=\textbf\def\PY@tc##1{\textcolor[rgb]{0.82,0.25,0.23}{##1}}}
\expandafter\def\csname PY@tok@nv\endcsname{\def\PY@tc##1{\textcolor[rgb]{0.10,0.09,0.49}{##1}}}
\expandafter\def\csname PY@tok@no\endcsname{\def\PY@tc##1{\textcolor[rgb]{0.53,0.00,0.00}{##1}}}
\expandafter\def\csname PY@tok@nl\endcsname{\def\PY@tc##1{\textcolor[rgb]{0.63,0.63,0.00}{##1}}}
\expandafter\def\csname PY@tok@ni\endcsname{\let\PY@bf=\textbf\def\PY@tc##1{\textcolor[rgb]{0.60,0.60,0.60}{##1}}}
\expandafter\def\csname PY@tok@na\endcsname{\def\PY@tc##1{\textcolor[rgb]{0.49,0.56,0.16}{##1}}}
\expandafter\def\csname PY@tok@nt\endcsname{\let\PY@bf=\textbf\def\PY@tc##1{\textcolor[rgb]{0.00,0.50,0.00}{##1}}}
\expandafter\def\csname PY@tok@nd\endcsname{\def\PY@tc##1{\textcolor[rgb]{0.67,0.13,1.00}{##1}}}
\expandafter\def\csname PY@tok@s\endcsname{\def\PY@tc##1{\textcolor[rgb]{0.73,0.13,0.13}{##1}}}
\expandafter\def\csname PY@tok@sd\endcsname{\let\PY@it=\textit\def\PY@tc##1{\textcolor[rgb]{0.73,0.13,0.13}{##1}}}
\expandafter\def\csname PY@tok@si\endcsname{\let\PY@bf=\textbf\def\PY@tc##1{\textcolor[rgb]{0.73,0.40,0.53}{##1}}}
\expandafter\def\csname PY@tok@se\endcsname{\let\PY@bf=\textbf\def\PY@tc##1{\textcolor[rgb]{0.73,0.40,0.13}{##1}}}
\expandafter\def\csname PY@tok@sr\endcsname{\def\PY@tc##1{\textcolor[rgb]{0.73,0.40,0.53}{##1}}}
\expandafter\def\csname PY@tok@ss\endcsname{\def\PY@tc##1{\textcolor[rgb]{0.10,0.09,0.49}{##1}}}
\expandafter\def\csname PY@tok@sx\endcsname{\def\PY@tc##1{\textcolor[rgb]{0.00,0.50,0.00}{##1}}}
\expandafter\def\csname PY@tok@m\endcsname{\def\PY@tc##1{\textcolor[rgb]{0.40,0.40,0.40}{##1}}}
\expandafter\def\csname PY@tok@gh\endcsname{\let\PY@bf=\textbf\def\PY@tc##1{\textcolor[rgb]{0.00,0.00,0.50}{##1}}}
\expandafter\def\csname PY@tok@gu\endcsname{\let\PY@bf=\textbf\def\PY@tc##1{\textcolor[rgb]{0.50,0.00,0.50}{##1}}}
\expandafter\def\csname PY@tok@gd\endcsname{\def\PY@tc##1{\textcolor[rgb]{0.63,0.00,0.00}{##1}}}
\expandafter\def\csname PY@tok@gi\endcsname{\def\PY@tc##1{\textcolor[rgb]{0.00,0.63,0.00}{##1}}}
\expandafter\def\csname PY@tok@gr\endcsname{\def\PY@tc##1{\textcolor[rgb]{1.00,0.00,0.00}{##1}}}
\expandafter\def\csname PY@tok@ge\endcsname{\let\PY@it=\textit}
\expandafter\def\csname PY@tok@gs\endcsname{\let\PY@bf=\textbf}
\expandafter\def\csname PY@tok@gp\endcsname{\let\PY@bf=\textbf\def\PY@tc##1{\textcolor[rgb]{0.00,0.00,0.50}{##1}}}
\expandafter\def\csname PY@tok@go\endcsname{\def\PY@tc##1{\textcolor[rgb]{0.53,0.53,0.53}{##1}}}
\expandafter\def\csname PY@tok@gt\endcsname{\def\PY@tc##1{\textcolor[rgb]{0.00,0.27,0.87}{##1}}}
\expandafter\def\csname PY@tok@err\endcsname{\def\PY@bc##1{\setlength{\fboxsep}{0pt}\fcolorbox[rgb]{1.00,0.00,0.00}{1,1,1}{\strut ##1}}}
\expandafter\def\csname PY@tok@kc\endcsname{\let\PY@bf=\textbf\def\PY@tc##1{\textcolor[rgb]{0.00,0.50,0.00}{##1}}}
\expandafter\def\csname PY@tok@kd\endcsname{\let\PY@bf=\textbf\def\PY@tc##1{\textcolor[rgb]{0.00,0.50,0.00}{##1}}}
\expandafter\def\csname PY@tok@kn\endcsname{\let\PY@bf=\textbf\def\PY@tc##1{\textcolor[rgb]{0.00,0.50,0.00}{##1}}}
\expandafter\def\csname PY@tok@kr\endcsname{\let\PY@bf=\textbf\def\PY@tc##1{\textcolor[rgb]{0.00,0.50,0.00}{##1}}}
\expandafter\def\csname PY@tok@bp\endcsname{\def\PY@tc##1{\textcolor[rgb]{0.00,0.50,0.00}{##1}}}
\expandafter\def\csname PY@tok@fm\endcsname{\def\PY@tc##1{\textcolor[rgb]{0.00,0.00,1.00}{##1}}}
\expandafter\def\csname PY@tok@vc\endcsname{\def\PY@tc##1{\textcolor[rgb]{0.10,0.09,0.49}{##1}}}
\expandafter\def\csname PY@tok@vg\endcsname{\def\PY@tc##1{\textcolor[rgb]{0.10,0.09,0.49}{##1}}}
\expandafter\def\csname PY@tok@vi\endcsname{\def\PY@tc##1{\textcolor[rgb]{0.10,0.09,0.49}{##1}}}
\expandafter\def\csname PY@tok@vm\endcsname{\def\PY@tc##1{\textcolor[rgb]{0.10,0.09,0.49}{##1}}}
\expandafter\def\csname PY@tok@sa\endcsname{\def\PY@tc##1{\textcolor[rgb]{0.73,0.13,0.13}{##1}}}
\expandafter\def\csname PY@tok@sb\endcsname{\def\PY@tc##1{\textcolor[rgb]{0.73,0.13,0.13}{##1}}}
\expandafter\def\csname PY@tok@sc\endcsname{\def\PY@tc##1{\textcolor[rgb]{0.73,0.13,0.13}{##1}}}
\expandafter\def\csname PY@tok@dl\endcsname{\def\PY@tc##1{\textcolor[rgb]{0.73,0.13,0.13}{##1}}}
\expandafter\def\csname PY@tok@s2\endcsname{\def\PY@tc##1{\textcolor[rgb]{0.73,0.13,0.13}{##1}}}
\expandafter\def\csname PY@tok@sh\endcsname{\def\PY@tc##1{\textcolor[rgb]{0.73,0.13,0.13}{##1}}}
\expandafter\def\csname PY@tok@s1\endcsname{\def\PY@tc##1{\textcolor[rgb]{0.73,0.13,0.13}{##1}}}
\expandafter\def\csname PY@tok@mb\endcsname{\def\PY@tc##1{\textcolor[rgb]{0.40,0.40,0.40}{##1}}}
\expandafter\def\csname PY@tok@mf\endcsname{\def\PY@tc##1{\textcolor[rgb]{0.40,0.40,0.40}{##1}}}
\expandafter\def\csname PY@tok@mh\endcsname{\def\PY@tc##1{\textcolor[rgb]{0.40,0.40,0.40}{##1}}}
\expandafter\def\csname PY@tok@mi\endcsname{\def\PY@tc##1{\textcolor[rgb]{0.40,0.40,0.40}{##1}}}
\expandafter\def\csname PY@tok@il\endcsname{\def\PY@tc##1{\textcolor[rgb]{0.40,0.40,0.40}{##1}}}
\expandafter\def\csname PY@tok@mo\endcsname{\def\PY@tc##1{\textcolor[rgb]{0.40,0.40,0.40}{##1}}}
\expandafter\def\csname PY@tok@ch\endcsname{\let\PY@it=\textit\def\PY@tc##1{\textcolor[rgb]{0.25,0.50,0.50}{##1}}}
\expandafter\def\csname PY@tok@cm\endcsname{\let\PY@it=\textit\def\PY@tc##1{\textcolor[rgb]{0.25,0.50,0.50}{##1}}}
\expandafter\def\csname PY@tok@cpf\endcsname{\let\PY@it=\textit\def\PY@tc##1{\textcolor[rgb]{0.25,0.50,0.50}{##1}}}
\expandafter\def\csname PY@tok@c1\endcsname{\let\PY@it=\textit\def\PY@tc##1{\textcolor[rgb]{0.25,0.50,0.50}{##1}}}
\expandafter\def\csname PY@tok@cs\endcsname{\let\PY@it=\textit\def\PY@tc##1{\textcolor[rgb]{0.25,0.50,0.50}{##1}}}

\def\PYZbs{\char`\\}
\def\PYZus{\char`\_}
\def\PYZob{\char`\{}
\def\PYZcb{\char`\}}
\def\PYZca{\char`\^}
\def\PYZam{\char`\&}
\def\PYZlt{\char`\<}
\def\PYZgt{\char`\>}
\def\PYZsh{\char`\#}
\def\PYZpc{\char`\%}
\def\PYZdl{\char`\$}
\def\PYZhy{\char`\-}
\def\PYZsq{\char`\'}
\def\PYZdq{\char`\"}
\def\PYZti{\char`\~}
% for compatibility with earlier versions
\def\PYZat{@}
\def\PYZlb{[}
\def\PYZrb{]}
\makeatother


    % Exact colors from NB
    \definecolor{incolor}{rgb}{0.0, 0.0, 0.5}
    \definecolor{outcolor}{rgb}{0.545, 0.0, 0.0}



    
    % Prevent overflowing lines due to hard-to-break entities
    \sloppy 
    % Setup hyperref package
    \hypersetup{
      breaklinks=true,  % so long urls are correctly broken across lines
      colorlinks=true,
      urlcolor=urlcolor,
      linkcolor=linkcolor,
      citecolor=citecolor,
      }
    % Slightly bigger margins than the latex defaults
    
    \geometry{verbose,tmargin=1in,bmargin=1in,lmargin=1in,rmargin=1in}
    
    

    \begin{document}
    
    
    \maketitle
    
    

    
    \hypertarget{chris-daigle}{%
\section{Chris Daigle}\label{chris-daigle}}

\hypertarget{convex-optimization-w-python}{%
\subsection{Convex Optimization w/
Python}\label{convex-optimization-w-python}}

\hypertarget{exam}{%
\subsection{Exam}\label{exam}}

    \begin{Verbatim}[commandchars=\\\{\}]
{\color{incolor}In [{\color{incolor}1}]:} \PY{k+kn}{import} \PY{n+nn}{pandas} \PY{k}{as} \PY{n+nn}{pd}
        \PY{k+kn}{import} \PY{n+nn}{numpy} \PY{k}{as} \PY{n+nn}{np}
        \PY{k+kn}{import} \PY{n+nn}{matplotlib}\PY{n+nn}{.}\PY{n+nn}{pyplot} \PY{k}{as} \PY{n+nn}{plt}
        \PY{k+kn}{import} \PY{n+nn}{seaborn} \PY{k}{as} \PY{n+nn}{sns}
        \PY{k+kn}{from} \PY{n+nn}{sympy} \PY{k}{import} \PY{o}{*}
        \PY{k+kn}{from} \PY{n+nn}{cvxpy} \PY{k}{import} \PY{o}{*}
\end{Verbatim}


    \begin{Verbatim}[commandchars=\\\{\}]
{\color{incolor}In [{\color{incolor}2}]:} \PY{n}{df} \PY{o}{=} \PY{n}{pd}\PY{o}{.}\PY{n}{read\PYZus{}csv}\PY{p}{(}
            \PY{l+s+s1}{\PYZsq{}}\PY{l+s+s1}{/Users/daiglechris/Git/ConvexOptimizationWithPython/Data/Stocks2\PYZhy{}closeP.csv}\PY{l+s+s1}{\PYZsq{}}\PY{p}{,} 
            \PY{n}{sep} \PY{o}{=} \PY{l+s+s2}{\PYZdq{}}\PY{l+s+s2}{,}\PY{l+s+s2}{\PYZdq{}}\PY{p}{)}
\end{Verbatim}


    \begin{Verbatim}[commandchars=\\\{\}]
{\color{incolor}In [{\color{incolor}3}]:} \PY{n}{df}\PY{o}{.}\PY{n}{head}\PY{p}{(}\PY{p}{)}
\end{Verbatim}


\begin{Verbatim}[commandchars=\\\{\}]
{\color{outcolor}Out[{\color{outcolor}3}]:}     timestamp  F\_close  BA\_close  RTN\_close  VLO\_close  AAL\_close  CI\_close  \textbackslash{}
        0  2018-12-03     9.60    359.96     172.87      81.30      39.65    224.84   
        1  2018-11-30     9.41    346.76     175.34      79.90      40.16    223.38   
        2  2018-11-29     9.37    342.56     173.58      79.47      38.42    222.52   
        3  2018-11-28     9.41    333.50     173.61      79.42      38.94    221.72   
        4  2018-11-27     9.28    318.03     171.67      77.43      38.29    216.80   
        
           IBM\_close  PG\_close  
        0     125.31     93.32  
        1     124.27     94.51  
        2     121.48     92.82  
        3     123.00     93.01  
        4     120.03     92.72  
\end{Verbatim}
            
    \hypertarget{question-1}{%
\subsubsection{Question 1:}\label{question-1}}

\begin{enumerate}
\def\labelenumi{(\alph{enumi})}
\tightlist
\item
  Compute the one day returns
  \(\left (\dfrac{p_{t+1}-p_{t}}{p_{t}} \right )\) of each stock and
  find the mean and the variance of each stock's returns.
\end{enumerate}

    \begin{Verbatim}[commandchars=\\\{\}]
{\color{incolor}In [{\color{incolor}4}]:} \PY{n}{oneDayReturns} \PY{o}{=} \PY{n}{df}\PY{o}{.}\PY{n}{iloc}\PY{p}{[}\PY{p}{:}\PY{p}{,}\PY{l+m+mi}{1}\PY{p}{:}\PY{p}{]}\PY{o}{.}\PY{n}{apply}\PY{p}{(}\PY{k}{lambda} \PY{n}{x}\PY{p}{:} \PY{p}{(}\PY{p}{(}\PY{n}{x} \PY{o}{\PYZhy{}} \PY{p}{(}\PY{n}{x}\PY{o}{.}\PY{n}{shift}\PY{p}{(}\PY{l+m+mi}{1}\PY{p}{)}\PY{p}{)}\PY{p}{)} \PY{o}{/} \PY{p}{(}\PY{n}{x}\PY{o}{.}\PY{n}{shift}\PY{p}{(}\PY{l+m+mi}{1}\PY{p}{)}\PY{p}{)}\PY{p}{)}\PY{p}{)}
        \PY{n}{names} \PY{o}{=} \PY{n+nb}{list}\PY{p}{(}\PY{n}{oneDayReturns}\PY{o}{.}\PY{n}{columns}\PY{p}{)}
        \PY{n}{meansOfReturns} \PY{o}{=} \PY{n}{pd}\PY{o}{.}\PY{n}{DataFrame}\PY{p}{(}\PY{n}{oneDayReturns}\PY{o}{.}\PY{n}{describe}\PY{p}{(}\PY{p}{)}\PY{o}{.}\PY{n}{iloc}\PY{p}{[}\PY{l+m+mi}{1}\PY{p}{]}\PY{o}{.}\PY{n}{abs}\PY{p}{(}\PY{p}{)}\PY{p}{,}
                                      \PY{n}{index} \PY{o}{=} \PY{n}{names}\PY{p}{)}
        \PY{n}{meansOfReturns1} \PY{o}{=} \PY{n+nb}{abs}\PY{p}{(}\PY{n}{oneDayReturns}\PY{o}{.}\PY{n}{describe}\PY{p}{(}\PY{p}{)}\PY{o}{.}\PY{n}{iloc}\PY{p}{[}\PY{l+m+mi}{1}\PY{p}{]}\PY{o}{.}\PY{n}{values}\PY{p}{)}
        \PY{n}{varianceOfReturns} \PY{o}{=} \PY{n}{oneDayReturns}\PY{o}{.}\PY{n}{var}\PY{p}{(}\PY{p}{)}
        \PY{n}{table} \PY{o}{=} \PY{n}{pd}\PY{o}{.}\PY{n}{concat}\PY{p}{(}\PY{p}{[}\PY{n}{meansOfReturns}\PY{p}{,} \PY{n}{varianceOfReturns}\PY{p}{]}\PY{p}{,}
                          \PY{n}{axis} \PY{o}{=} \PY{l+m+mi}{1}\PY{p}{)}
        \PY{n}{table}\PY{o}{.}\PY{n}{columns} \PY{o}{=} \PY{p}{[}\PY{l+s+s1}{\PYZsq{}}\PY{l+s+s1}{Mean of the Returns}\PY{l+s+s1}{\PYZsq{}}\PY{p}{,} \PY{l+s+s1}{\PYZsq{}}\PY{l+s+s1}{Variance of the Returns}\PY{l+s+s1}{\PYZsq{}}\PY{p}{]}
        \PY{n}{table}
\end{Verbatim}


\begin{Verbatim}[commandchars=\\\{\}]
{\color{outcolor}Out[{\color{outcolor}4}]:}            Mean of the Returns  Variance of the Returns
        F\_close               0.001533                 0.000352
        BA\_close              0.000098                 0.000331
        RTN\_close             0.001566                 0.000214
        VLO\_close             0.003054                 0.000501
        AAL\_close             0.000345                 0.000649
        CI\_close              0.002542                 0.000194
        IBM\_close             0.001668                 0.000267
        PG\_close              0.001568                 0.000145
\end{Verbatim}
            
    \begin{longtable}[]{@{}lcr@{}}
\toprule
\textbf{Name of Stock} & \textbf{Mean of the Returns} & \textbf{Variance
of the Returns}\tabularnewline
\midrule
\endhead
Ford & 0.001533 & 0.000352\tabularnewline
Boeing & 0.000098 & 0.000331\tabularnewline
Raytheon & 0.001566 & 0.000214\tabularnewline
Valero Energy & 0.003054 & 0.000501\tabularnewline
American Airlines & 0.000345 & 0.000649\tabularnewline
Cigna & 0.002542 & 0.000194\tabularnewline
IBM & 0.001668 & 0.000267\tabularnewline
Procter \& Gamble & 0.001568 & 0.000145\tabularnewline
\bottomrule
\end{longtable}

    \begin{enumerate}
\def\labelenumi{(\alph{enumi})}
\setcounter{enumi}{1}
\tightlist
\item
  Plot the returns of the stocks with the two highest variance and the
  stocks with the two highest mean. Save a pdf copy of you graphs with
  ``plt.savefig(ʼsample.pdfʼ). Send the pdf file to
  olivier.morand@uconn.edu with ``graph-your name'' in the title of your
  email.
\end{enumerate}

    \begin{Verbatim}[commandchars=\\\{\}]
{\color{incolor}In [{\color{incolor}5}]:} \PY{n}{highestVar} \PY{o}{=} \PY{n}{table}\PY{o}{.}\PY{n}{sort\PYZus{}values}\PY{p}{(}\PY{n}{by} \PY{o}{=} \PY{l+s+s1}{\PYZsq{}}\PY{l+s+s1}{Variance of the Returns}\PY{l+s+s1}{\PYZsq{}}\PY{p}{,}
                                       \PY{n}{axis} \PY{o}{=} \PY{l+m+mi}{0}\PY{p}{,}
                                       \PY{n}{ascending}\PY{o}{=}\PY{k+kc}{False}\PY{p}{)}\PY{o}{.}\PY{n}{head}\PY{p}{(}\PY{l+m+mi}{2}\PY{p}{)}
        \PY{n}{highestMean} \PY{o}{=} \PY{n}{table}\PY{o}{.}\PY{n}{sort\PYZus{}values}\PY{p}{(}\PY{n}{by} \PY{o}{=} \PY{l+s+s1}{\PYZsq{}}\PY{l+s+s1}{Mean of the Returns}\PY{l+s+s1}{\PYZsq{}}\PY{p}{,}
                                        \PY{n}{axis} \PY{o}{=} \PY{l+m+mi}{0}\PY{p}{,}
                                        \PY{n}{ascending}\PY{o}{=}\PY{k+kc}{False}\PY{p}{)}\PY{o}{.}\PY{n}{head}\PY{p}{(}\PY{l+m+mi}{2}\PY{p}{)}
        \PY{n}{dfHighest} \PY{o}{=} \PY{n}{pd}\PY{o}{.}\PY{n}{concat}\PY{p}{(}\PY{p}{[}\PY{n}{highestMean}\PY{p}{,} \PY{n}{highestVar}\PY{p}{]}\PY{p}{)}
        \PY{n}{dfReturnHighest} \PY{o}{=} \PY{n}{oneDayReturns}\PY{p}{[}\PY{n}{dfHighest}\PY{o}{.}\PY{n}{T}\PY{o}{.}\PY{n}{columns}\PY{p}{]}
\end{Verbatim}


    \begin{Verbatim}[commandchars=\\\{\}]
{\color{incolor}In [{\color{incolor}6}]:} \PY{n}{dfReturnHighest}\PY{o}{.}\PY{n}{plot}\PY{p}{(}\PY{n}{grid} \PY{o}{=} \PY{k+kc}{True}\PY{p}{,} 
                             \PY{n}{figsize} \PY{o}{=} \PY{p}{(}\PY{l+m+mi}{12}\PY{p}{,}\PY{l+m+mi}{8}\PY{p}{)}\PY{p}{)}\PY{o}{.}\PY{n}{axhline}\PY{p}{(}\PY{n}{y} \PY{o}{=} \PY{l+m+mi}{0}\PY{p}{,} 
                                                       \PY{n}{color} \PY{o}{=} \PY{l+s+s2}{\PYZdq{}}\PY{l+s+s2}{black}\PY{l+s+s2}{\PYZdq{}}\PY{p}{,} 
                                                       \PY{n}{lw} \PY{o}{=} \PY{l+m+mi}{2}\PY{p}{)}
        \PY{n}{plt}\PY{o}{.}\PY{n}{title}\PY{p}{(}\PY{l+s+s1}{\PYZsq{}}\PY{l+s+s1}{Returns of the stocks with the two highest variance and the stocks with the two highest mean}\PY{l+s+s1}{\PYZsq{}}\PY{p}{)}
        \PY{n}{plt}\PY{o}{.}\PY{n}{savefig}\PY{p}{(}\PY{l+s+s1}{\PYZsq{}}\PY{l+s+s1}{graph\PYZhy{}Daigle.pdf}\PY{l+s+s1}{\PYZsq{}}\PY{p}{)}
        \PY{n}{plt}\PY{o}{.}\PY{n}{show}\PY{p}{(}\PY{p}{)}
\end{Verbatim}


    \begin{center}
    \adjustimage{max size={0.9\linewidth}{0.9\paperheight}}{output_9_0.png}
    \end{center}
    { \hspace*{\fill} \\}
    
    \hypertarget{question-2}{%
\subsubsection{Question 2:}\label{question-2}}

Two clients, John and Jane, have asked you for help to build a portfolio
in which to invest their income. You seek a combination of the 8 stocks
that will maximize each clientʼs utility. A portfolio is a vector
\(X = (x_{1}, ..., x_{8})\) with \(\forall i = 1, ... 8\),
\(0 \le x_{i} \le 1\) and \(\sum_{i=1}^{8} x_{i} = 1\) where \(x_{i}\)
is the fraction of income to be spend on asset \(i\). Both clients have
utility: \[X.\overline{R}^{T} - \frac{1}{2}AX\Sigma X^{T}\] where
\(\overline{R}\) is the \(1 \times 8\) vector of means calculated in 1,
and \(\Sigma\) is the \(8 \times 8\) covariance matrix of returns. John
is risk averse and his risk coefficient is \(A = 2\) while Jane's
coefficient is \(A = 10\).

    \begin{enumerate}
\def\labelenumi{(\alph{enumi})}
\tightlist
\item
  Explain below why the problem is convex.
\end{enumerate}

The problem is convex because the Hessian Matrix of \(X\Sigma X^{T}\) is
positive semidefinite (PSD), \(\frac{1}{2} A\) only scales the value of
\(X\Sigma X^{T}\), and the first portion \(X.\overline{R}^{T}\) is just
a vector of scalars, singleton or otherwise. So, we only need to
evaluate the Hessian of \(X\Sigma X^{T}\).

    \begin{Verbatim}[commandchars=\\\{\}]
{\color{incolor}In [{\color{incolor}7}]:} \PY{n}{x1}\PY{p}{,} \PY{n}{x2}\PY{p}{,} \PY{n}{x3}\PY{p}{,} \PY{n}{x4}\PY{p}{,} \PY{n}{x5}\PY{p}{,} \PY{n}{x6}\PY{p}{,} \PY{n}{x7}\PY{p}{,} \PY{n}{x8} \PY{o}{=} \PY{n}{symbols}\PY{p}{(}\PY{l+s+s1}{\PYZsq{}}\PY{l+s+s1}{ford, Boeing, Raytheon, Valero, AmericanAir, Cigna, IBM, PG}\PY{l+s+s1}{\PYZsq{}}\PY{p}{)}
        \PY{n}{X} \PY{o}{=} \PY{n}{np}\PY{o}{.}\PY{n}{array}\PY{p}{(}\PY{p}{[}\PY{n}{x1}\PY{p}{,} \PY{n}{x2}\PY{p}{,} \PY{n}{x3}\PY{p}{,} \PY{n}{x4}\PY{p}{,} \PY{n}{x5}\PY{p}{,} \PY{n}{x6}\PY{p}{,} \PY{n}{x7}\PY{p}{,} \PY{n}{x8}\PY{p}{]}\PY{p}{)}
        \PY{n}{R} \PY{o}{=} \PY{n}{meansOfReturns}
        \PY{n}{Sigma} \PY{o}{=} \PY{n}{oneDayReturns}\PY{o}{.}\PY{n}{cov}\PY{p}{(}\PY{p}{)}
        \PY{n}{Z} \PY{o}{=} \PY{n}{np}\PY{o}{.}\PY{n}{array}\PY{p}{(}\PY{n}{X}\PY{o}{*}\PY{n}{Sigma}\PY{o}{*}\PY{n}{X}\PY{o}{.}\PY{n}{T}\PY{p}{)}
\end{Verbatim}


    \begin{Verbatim}[commandchars=\\\{\}]
{\color{incolor}In [{\color{incolor}8}]:} \PY{k}{def} \PY{n+nf}{hessian}\PY{p}{(}\PY{n}{x}\PY{p}{)}\PY{p}{:}
            \PY{l+s+sd}{\PYZdq{}\PYZdq{}\PYZdq{}}
        \PY{l+s+sd}{    Calculate the hessian matrix with finite differences}
        \PY{l+s+sd}{    Parameters:}
        \PY{l+s+sd}{       \PYZhy{} x : ndarray}
        \PY{l+s+sd}{    Returns:}
        \PY{l+s+sd}{       an array of shape (x.dim, x.ndim) + x.shape}
        \PY{l+s+sd}{       where the array[i, j, ...] corresponds to the second derivative x\PYZus{}ij}
        \PY{l+s+sd}{    \PYZdq{}\PYZdq{}\PYZdq{}}
            \PY{n}{x\PYZus{}grad} \PY{o}{=} \PY{n}{np}\PY{o}{.}\PY{n}{gradient}\PY{p}{(}\PY{n}{x}\PY{p}{)} 
            \PY{n}{hessian} \PY{o}{=} \PY{n}{np}\PY{o}{.}\PY{n}{empty}\PY{p}{(}\PY{p}{(}\PY{n}{x}\PY{o}{.}\PY{n}{ndim}\PY{p}{,} \PY{n}{x}\PY{o}{.}\PY{n}{ndim}\PY{p}{)} \PY{o}{+} \PY{n}{x}\PY{o}{.}\PY{n}{shape}\PY{p}{,} \PY{n}{dtype}\PY{o}{=}\PY{n}{x}\PY{o}{.}\PY{n}{dtype}\PY{p}{)} 
            \PY{k}{for} \PY{n}{k}\PY{p}{,} \PY{n}{grad\PYZus{}k} \PY{o+ow}{in} \PY{n+nb}{enumerate}\PY{p}{(}\PY{n}{x\PYZus{}grad}\PY{p}{)}\PY{p}{:}
                \PY{c+c1}{\PYZsh{} iterate over dimensions}
                \PY{c+c1}{\PYZsh{} apply gradient again to every component of the first derivative.}
                \PY{n}{tmp\PYZus{}grad} \PY{o}{=} \PY{n}{np}\PY{o}{.}\PY{n}{gradient}\PY{p}{(}\PY{n}{grad\PYZus{}k}\PY{p}{)} 
                \PY{k}{for} \PY{n}{l}\PY{p}{,} \PY{n}{grad\PYZus{}kl} \PY{o+ow}{in} \PY{n+nb}{enumerate}\PY{p}{(}\PY{n}{tmp\PYZus{}grad}\PY{p}{)}\PY{p}{:}
                    \PY{n}{hessian}\PY{p}{[}\PY{n}{k}\PY{p}{,} \PY{n}{l}\PY{p}{,} \PY{p}{:}\PY{p}{,} \PY{p}{:}\PY{p}{]} \PY{o}{=} \PY{n}{grad\PYZus{}kl}
            \PY{k}{return} \PY{n}{hessian}
\end{Verbatim}


    \begin{Verbatim}[commandchars=\\\{\}]
{\color{incolor}In [{\color{incolor}9}]:} \PY{n}{H} \PY{o}{=} \PY{n}{hessian}\PY{p}{(}\PY{n}{np}\PY{o}{.}\PY{n}{array}\PY{p}{(}\PY{n}{Sigma}\PY{p}{)}\PY{p}{)}
        \PY{n}{np}\PY{o}{.}\PY{n}{linalg}\PY{o}{.}\PY{n}{det}\PY{p}{(}\PY{n}{H}\PY{p}{)}
\end{Verbatim}


\begin{Verbatim}[commandchars=\\\{\}]
{\color{outcolor}Out[{\color{outcolor}9}]:} array([[ 2.20799925e-63, -6.76560985e-48],
               [-7.95272270e-48, -3.57667386e-65]])
\end{Verbatim}
            
    This works out to a series of zeros in the matrix. I'm not sure why
there are 4 values, but regardless, zero times any real value is zero
and thus, the matrix is PSD and the problem is thus convex.

    \begin{enumerate}
\def\labelenumi{(\alph{enumi})}
\setcounter{enumi}{1}
\tightlist
\item
  Solve for Jane's and Johns' optimal portfolio. Write your answer below
\end{enumerate}

    \begin{Verbatim}[commandchars=\\\{\}]
{\color{incolor}In [{\color{incolor}10}]:} \PY{n}{mu} \PY{o}{=} \PY{n}{meansOfReturns1}
         
         \PY{n}{w} \PY{o}{=} \PY{n}{Variable}\PY{p}{(}\PY{l+m+mi}{8}\PY{p}{)} \PY{c+c1}{\PYZsh{} Decision variable}
         
         \PY{n}{gamma} \PY{o}{=} \PY{l+m+mi}{2} \PY{c+c1}{\PYZsh{} Risk Parameter in the utility function}
         \PY{n}{ret} \PY{o}{=} \PY{n}{mu}\PY{o}{*}\PY{n}{w}
         \PY{n}{risk} \PY{o}{=} \PY{n}{quad\PYZus{}form}\PY{p}{(}\PY{n}{w}\PY{p}{,} \PY{n}{Sigma}\PY{p}{)}
         
         \PY{n}{obj} \PY{o}{=} \PY{n}{Maximize}\PY{p}{(}\PY{n}{ret} \PY{o}{\PYZhy{}} \PY{n}{gamma}\PY{o}{*}\PY{n}{risk}\PY{p}{)}
         \PY{n}{constraints} \PY{o}{=} \PY{p}{[}\PY{n+nb}{sum}\PY{p}{(}\PY{n}{w}\PY{p}{)} \PY{o}{==} \PY{l+m+mi}{1} \PY{p}{,} \PY{n}{w} \PY{o}{\PYZgt{}}\PY{o}{=}\PY{l+m+mi}{0}\PY{p}{]}
         
         \PY{c+c1}{\PYZsh{} Form the problem}
         \PY{n}{prob} \PY{o}{=} \PY{n}{Problem}\PY{p}{(}\PY{n}{obj}\PY{p}{,} \PY{n}{constraints}\PY{p}{)}
\end{Verbatim}


    \begin{Verbatim}[commandchars=\\\{\}]
{\color{incolor}In [{\color{incolor}11}]:} \PY{n}{port\PYZus{}data} \PY{o}{=} \PY{p}{[}\PY{p}{]}
         \PY{n}{ret\PYZus{}data} \PY{o}{=} \PY{p}{[}\PY{p}{]}
         \PY{n}{risk\PYZus{}data} \PY{o}{=} \PY{p}{[}\PY{p}{]}
         \PY{n}{prob\PYZus{}data} \PY{o}{=} \PY{p}{[}\PY{p}{]}
         
         \PY{n}{gamma\PYZus{}vals} \PY{o}{=} \PY{p}{[}\PY{l+m+mi}{2}\PY{p}{,}\PY{l+m+mi}{10}\PY{p}{]}
         \PY{k}{for} \PY{n}{i} \PY{o+ow}{in} \PY{n+nb}{range}\PY{p}{(}\PY{l+m+mi}{2}\PY{p}{)}\PY{p}{:}
             \PY{n}{gamma} \PY{o}{=} \PY{n}{gamma\PYZus{}vals}\PY{p}{[}\PY{n}{i}\PY{p}{]}
             \PY{n}{prob}\PY{o}{.}\PY{n}{solve}\PY{p}{(}\PY{p}{)} \PY{c+c1}{\PYZsh{} Solve the problem for a specific value of gamma}
             \PY{n}{risk\PYZus{}data}\PY{o}{.}\PY{n}{append}\PY{p}{(}\PY{n}{sqrt}\PY{p}{(}\PY{n}{risk}\PY{p}{)}\PY{o}{.}\PY{n}{value}\PY{p}{)} \PY{c+c1}{\PYZsh{} optimal value of the risk (standard deviation)}
             \PY{n}{ret\PYZus{}data}\PY{o}{.}\PY{n}{append}\PY{p}{(}\PY{n}{ret}\PY{o}{.}\PY{n}{value}\PY{p}{)} \PY{c+c1}{\PYZsh{} optimal value of the return}
             \PY{n}{prob\PYZus{}data}\PY{o}{.}\PY{n}{append}\PY{p}{(}\PY{n}{prob}\PY{o}{.}\PY{n}{value}\PY{p}{)} \PY{c+c1}{\PYZsh{} maximum utility}
             \PY{n}{port\PYZus{}data}\PY{o}{.}\PY{n}{append}\PY{p}{(}\PY{n}{w}\PY{o}{.}\PY{n}{value}\PY{p}{)}  
\end{Verbatim}


    \begin{Verbatim}[commandchars=\\\{\}]
{\color{incolor}In [{\color{incolor}12}]:} \PY{n+nb}{print}\PY{p}{(}\PY{n}{np}\PY{o}{.}\PY{n}{sum}\PY{p}{(}\PY{n}{port\PYZus{}data}\PY{p}{[}\PY{l+m+mi}{0}\PY{p}{]}\PY{p}{)}\PY{p}{)}
         \PY{n+nb}{print}\PY{p}{(}\PY{n}{np}\PY{o}{.}\PY{n}{sum}\PY{p}{(}\PY{n}{port\PYZus{}data}\PY{p}{[}\PY{l+m+mi}{1}\PY{p}{]}\PY{p}{)}\PY{p}{)}
\end{Verbatim}


    \begin{Verbatim}[commandchars=\\\{\}]
1.0000000000000788
1.0000000000000788

    \end{Verbatim}

    Checking that the proportions add to 1. Because of floating point
arithmetic, the value is not exactly 1, but the results ``check out''

    \begin{Verbatim}[commandchars=\\\{\}]
{\color{incolor}In [{\color{incolor}13}]:} \PY{n}{results} \PY{o}{=} \PY{n}{pd}\PY{o}{.}\PY{n}{DataFrame}\PY{p}{(}
             \PY{p}{\PYZob{}}\PY{l+s+s2}{\PYZdq{}}\PY{l+s+s2}{John}\PY{l+s+s2}{\PYZsq{}}\PY{l+s+s2}{s Optimal Portfolio}\PY{l+s+s2}{\PYZdq{}}\PY{p}{:}\PY{n}{port\PYZus{}data}\PY{p}{[}\PY{l+m+mi}{0}\PY{p}{]}\PY{p}{,}
              \PY{l+s+s2}{\PYZdq{}}\PY{l+s+s2}{Jane}\PY{l+s+s2}{\PYZsq{}}\PY{l+s+s2}{s Optimal Portfolio}\PY{l+s+s2}{\PYZdq{}}\PY{p}{:}\PY{n}{port\PYZus{}data}\PY{p}{[}\PY{l+m+mi}{1}\PY{p}{]}\PY{p}{\PYZcb{}}\PY{p}{,}
             \PY{n}{index}\PY{o}{=} \PY{p}{[}\PY{l+s+s1}{\PYZsq{}}\PY{l+s+s1}{Ford}\PY{l+s+s1}{\PYZsq{}}\PY{p}{,} \PY{l+s+s1}{\PYZsq{}}\PY{l+s+s1}{Boeing}\PY{l+s+s1}{\PYZsq{}}\PY{p}{,} \PY{l+s+s1}{\PYZsq{}}\PY{l+s+s1}{Raytheon}\PY{l+s+s1}{\PYZsq{}}\PY{p}{,} 
                           \PY{l+s+s1}{\PYZsq{}}\PY{l+s+s1}{Valero}\PY{l+s+s1}{\PYZsq{}}\PY{p}{,} \PY{l+s+s1}{\PYZsq{}}\PY{l+s+s1}{American Airlines}\PY{l+s+s1}{\PYZsq{}}\PY{p}{,} 
                           \PY{l+s+s1}{\PYZsq{}}\PY{l+s+s1}{Cigna}\PY{l+s+s1}{\PYZsq{}}\PY{p}{,} \PY{l+s+s1}{\PYZsq{}}\PY{l+s+s1}{IBM}\PY{l+s+s1}{\PYZsq{}}\PY{p}{,} \PY{l+s+s1}{\PYZsq{}}\PY{l+s+s1}{ PG}\PY{l+s+s1}{\PYZsq{}}\PY{p}{]}\PY{p}{)}
         \PY{n}{results}
\end{Verbatim}


\begin{Verbatim}[commandchars=\\\{\}]
{\color{outcolor}Out[{\color{outcolor}13}]:}                    John's Optimal Portfolio  Jane's Optimal Portfolio
         Ford                           7.262696e-08              7.262696e-08
         Boeing                         2.046956e-08              2.046956e-08
         Raytheon                       6.250528e-08              6.250528e-08
         Valero                         4.409146e-01              4.409146e-01
         American Airlines              2.174878e-08              2.174878e-08
         Cigna                          5.590849e-01              5.590849e-01
         IBM                            7.084433e-08              7.084433e-08
          PG                            2.445910e-07              2.445910e-07
\end{Verbatim}
            

    % Add a bibliography block to the postdoc
    
    
    
    \end{document}
